% Temlate for an article written in the IEEE style.
\documentclass[journal]{IEEEtran}
% Enables the use of colour. 
\usepackage{xcolor}
% Syntax high-lighting for code. Requires Python's pygments.
\usepackage{minted}
% Enables the use of umlauts and other accents.
\usepackage[utf8]{inputenc}
% Diagrams.
\usepackage{tikz}
% Settings for captions, such as sideways captions.
\usepackage{caption}
% Symbols for units, like degrees and ohms.
\usepackage{gensymb}
% Latin modern fonts - better looking than the defaults.
\usepackage{lmodern}
% Allows for columns spanning multiple rows in tables.
\usepackage{multirow}
% Better looking tables, including nicer borders.
\usepackage{booktabs}
% More math symbols.
\usepackage{amssymb}
% More math fonts, like mathbb.
\usepackage{amsfonts}
% More math layouts, equation arrays, etc.
\usepackage{amsmath}
% More theorem environments.
\usepackage{amsthm}
% More column formats for tables.
\usepackage{array}
% Adjust the sizes of box environments.
\usepackage{adjustbox}
% Better looking single quotes in verbatim and minted environments.
\usepackage{upquote}
% Better blank space decisions.
\usepackage{xspace}
% Better looking tikz trees.
\usepackage{forest}
% URLs.
\usepackage{hyperref}
% Plotting.
\usepackage{pgfplots}
% Filler text.
\usepackage{lipsum}
% Line spacing.
\usepackage{setspace}
% Changing spacing on pages.
\usepackage{geometry}
% Gantt charts.
\usepackage{pgfgantt}


% Use latest pgfplots.
\pgfplotsset{compat=newest}

% Various tikz libraries.
% For drawing mind maps.
\usetikzlibrary{mindmap}
% For adding shadows.
\usetikzlibrary{shadows}
% Extra arrows tips.
\usetikzlibrary{arrows.meta}
% Old arrows.
\usetikzlibrary{arrows}
% Automata.
\usetikzlibrary{automata}
% For more positioning options.
\usetikzlibrary{positioning}
% Creating chains of nodes on a line.
\usetikzlibrary{chains}
% Fitting node to contain set of coordinates.
\usetikzlibrary{fit}
% Extra shapes for drawing.
\usetikzlibrary{shapes}
% For markings on paths.
\usetikzlibrary{decorations.markings}
% For advanced calculations.
\usetikzlibrary{calc}

% GMIT colours.
\definecolor{gmitblue}{RGB}{20,134,225}
\definecolor{gmitred}{RGB}{220,20,60}
\definecolor{gmitgrey}{RGB}{67,67,67}


% Tell minted to use the following colour scheme. 
\usemintedstyle{manni}
% Set some minted options.
\setminted{frame=lines, framesep=2mm, baselinestretch=1.2, linenos}


\begin{document}

\title{Example paper}

\author{Dr Ian McLoughlin (ian.mcloughlin@gmit.ie) \\ Galway-Mayo Institute of Technology}
\date{\today}

\maketitle


\begin{abstract}
  The abstract gives a brief overview of the main contribution of the paper.
  The important results in the paper should be front and centre here.
  Many people who arrive at your paper will only read the abstract, and those
  who do read it will likely decide to based on the abstract.
\end{abstract}

\begin{IEEEkeywords}
  IEEE, IEEEtran, journal, \LaTeX, paper, template.
\end{IEEEkeywords}


\section{Introduction}
  The introduction to the paper should expand on the abstract, give the reader
  an idea of what you assume they already know, and inform them of the layout of
  the paper.

\section{Literature}
  Sometimes a review of the relevant literature in an area is included in the
  introduction, but it is no harm to put it in its own section. The paper
  overall should be full of references to other papers, but this is particularly
  the case for the literature section.
  
  To find relevant literature, it's no harm to first have a general Google of
  the topic. However, you will eventually want to be systematic about it so
  that you don't have any obvious omissions of particularly relevant works.
  A good way to do this is to find a list of journals relevant to your topic,
  covering different publishers, decide on a list of search terms, and 
  one-by-one search through the journals. For the papers that are returned,
  I recommend skimming the abstracts to determine whether the paper is worth
  reading. Then you must read them.

  Following that, it's a good idea to keep an eye on collections like
   arXiv~\cite{arxiv:home}.

\section{Main sections}
  After the introduction and summary of the literature to date, you should start
  into the main contribution of your paper. How you organise that in sections
  is a matter of what kind of result it is. If you have run some experiments
  or your results are empirical, you might have a methodology section, followed
  by a results section.

  However, not every paper will have a methodology section. Generally speaking,
  if you are proving something outright, like showing that an algorithm has a
  certain property, you usually organise the sections in a bespoke manner suited
  to your result. You might, for instance, have an algorithm section describing
  the algorithm followed by a complexity section describing the computational
  complexity of the algorithm.

\section{Conclusion}
  The conclusion often just summarises what the paper has just said.

\bibliographystyle{ieeetr}
\bibliography{bibliography}

\end{document} 